\section{Practice Problems 4 - Recursion}
Practice Problem 4 are inspired from the following video do watch till 6:10 to get clear understanding of recursion
\href{https://youtu.be/ngCos392W4w}{5 Simple Steps for Solving Any Recursive Problem}
\begin{itemize}
	\item What's the simplest possible input?
	\item Play around with examples and visualize!
	\item Relate hard cases to simpler cases
	\item Generalize the pattern
	\item Write code by combining recursive pattern with base case
\end{itemize}
\begin{enumerate}
\item \textbf{Ackermann function}\\
Write a recursive function $\op{A}()$ that takes two inputs $n$ and $m$ and outputs the number $\op{A}(m,n)$ where $\op{A}(m,n)$  is defined as
\begin{align}
	\op{A}(0,n)&=n+1\\
	\op{A}(m+1,0)&=\op{A}(m,1)\\
	\op{A}(m+1,n+1)&=\op{A}(m,\op{A}(m+1,n))
\end{align}
\textbf{Problem Statement:}\\
For $k$ pairs $(m_1,n_1),(m_2,n_2),\ldots,(m_k,n_k)$, find $\op{A}(m,n)$ for such $m,n$\\
\begin{testcases}
	{$k$\\$m_i\ n_i$ (each pair on a newline)}
	{$\op{A}(m_i,n_i)$ (each result on a newline)}
	{7\\0 0\\0 4\\1 3\\2 2\\3 4\\4 0\\4 1}
	{1\\5\\5\\7\\125\\13\\65533}
\end{testcases}
\item \textbf{GridPaths}\\
Write a recursive function $\op{NumberOfGridPaths}()$ that takes two inputs $n$ and $m$ and outputs the number of unique paths from the top left corner to bottom right corner of a $n \times m$ grid.

Constraints: $n,m\geq1$ and you can only move down or right 1 unit at a time.\\
Examples given in Figure \ref{fig:gdp}\\
\begin{figure}[h!]
	\centering
	\begin{subfigure}{0.55\linewidth}
		\centering
		\includegraphics[width=\linewidth]{GridPaths.png}
		\caption{Number of Grid Paths for $m,n\in\{1,2,3\}$}
		\label{fig:gdp}
	\end{subfigure}
	\vline
	\begin{subfigure}{0.4\linewidth}
		\centering
		\includegraphics[width=\linewidth]{Delannoy3x3.pdf}
		\caption{DelannoyNumber(3,3) = 63}
		\label{fig:dela}
	\end{subfigure}
	\caption{}
\end{figure}
\textbf{Problem Statement:}\\
For $k$ pairs $(n_1,m_1),(n_2,m_2),\ldots,(n_k,m_k)$, find $\op{NumberOfGridPaths}(n,m)$ for such $n,m$\\%\verb!NumberOfGridPaths(n,m)! for such \verb!n,m!
\begin{testcases}
	{$k$\\$n_i\ m_i$ (each pair on a newline)}
	{$\op{NumberOfGridPaths}(n_i,m_i)$ (each result on a newline)}
	{6\\1 1\\2 5\\3 3\\6 3\\7 10\\17 8}
	{1\\5\\6\\21\\5005\\245157}
\end{testcases}
Can you find efficient solution?
\item \textbf{Delannoy Numbers}\\
Delannoy numbers describes the number of paths from the southwest corner (0, 0) of a rectangular grid to the northeast corner $(m, n)$, using only single steps north, northeast, or east.\\
Write a recursive function  $\op{DelannoyNumber}()$ to count number of these paths
Constraints: $n,m\geq0$\\
Examples given in Figure \ref{fig:dela}

\textbf{Problem Statement:}\\
For $k$ pairs $(n_1,m_1),(n_2,m_2),\ldots,(n_k,m_k)$, find $\op{DelannoyNumbers}(n,m)$ for such $n,m$\\
\begin{testcases}
	{$k$\\$n_i\ m_i$ (each pair on a newline)}
	{$\op{DelannoyNumber}(n_i,m_i)$ (each result on a newline)}
	{6\\1 1\\2 5\\3 3\\6 3\\7 10\\17 8}
	{3\\61\\63\\377\\433905\\245157\\62390545}
\end{testcases}
Can you find efficient solution?
\item \textbf{Partitions}
\begin{enumerate}[label=(\alph*)]
\item 
Write a recursive function $\op{NumberOfPartitions}()$ that counts the number of ways you can partition $n$ objects using parts up to $m$

Constraints: $n,m>0$\\
Examples given in Figure \ref{fig:part}
\begin{figure}[H]
	\centering
	\includegraphics[width=\linewidth]{Partitions.png}
	\caption{Partition $n$ objects using parts up to $m$}
	\label{fig:part}
\end{figure}
Can you find efficient solution?\\
\textbf{Problem Statement:}\\
For $k$ pairs $(n_1,m_1),(n_2,m_2),\ldots,(n_k,m_k)$, find $\op{NumberOfPartitions}(n,m)$ for such $n,m$\\
\begin{testcases}
	{$k$\\$n_i\ m_i$ (each pair on a newline)}
	{$\op{NumberOfPartitions}(n_i,m_i)$ (each result on a newline)}
	{7\\1 1\\2 5\\3 3\\6 3\\7 10\\17 8\\20 12}
	{1\\2\\3\\7\\15\\230\\582}
\end{testcases}
\item Number of partitions $\op{P}(n)$ of an integer $n$ is same as $\op{NumberOfPartitions}(n,n)$
\begin{thm}[Pentagonal Number Theorem]
	This theorem relates the product and series representations of the \href{https://en.wikipedia.org/wiki/Euler_function}{Euler function}
	\begin{equation}
		\prod_{n=1}^{\infty}\left(1-x^{n}\right)=\sum_{k=-\infty }^{\infty }\left(-1\right)^{k}x^{k\left(3k-1\right)/2}=1+\sum _{k=1}^{\infty }(-1)^{k}\left(x^{k(3k+1)/2}+x^{k(3k-1)/2}\right)
	\end{equation}
	In other words,
	\begin{equation}{\label{eqn:pnts}}
		(1-x)(1-x^{2})(1-x^{3})\cdots =1-x-x^{2}+x^{5}+x^{7}-x^{12}-x^{15}+x^{22}+x^{26}-\cdots
	\end{equation}
	The exponents $1, 2, 5, 7, 12,\ldots$ on the right hand side are called (generalized) pentagonal numbers (\href{https://oeis.org/A001318}{A001318}) and are given by the formula $g_k = k(3k - 1)/2$ for $k = 1, -1, 2, -2, 3,-3,\ldots$
\end{thm}
The equation (\ref{eqn:pnts}) implies a recurrence for calculating $\op{P}(n)$, the number of partitions of n:
\begin{equation}{\label{eqn:noprf}}
	\op{P}(n)=\op{P}(n-1)+\op{P}(n-2)-\op{P}(n-5)-\op{P}(n-7)+\cdots
\end{equation}
or more formally,
\begin{equation}
	\op{P}(n)=\sum_{k\neq 0}(-1)^{k-1}\op{P}(n-g_{k})
\end{equation}
\textbf{Problem Statement:}\\
Write a function $\op{P}()$ which calculates number of partitions using equation (\ref{eqn:noprf})\\
\begin{testcases}
	{Arbitrary number of testcases (each space separated)\\Stop when input is negative}
	{Number of partitions for each testcase (each on a newline)}
	{1 2 4 8 16 32 64 128 -1}
	{1\\2\\5\\22\\231\\8349\\1741630\\4351078600}
\end{testcases}
\subsection*{Crazy Video}
\href{https://youtu.be/iJ8pnCO0nTY}{The hardest What comes next (Euler's pentagonal formula)}
\subsection*{Exciting Puzzle}
\href{https://youtu.be/rf6uf3jNjbo}{Towers of Hanoi: A Complete Recursive Visualization}
\end{enumerate}
\end{enumerate}