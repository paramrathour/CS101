\section{Practice Problems 5 - Arrays}
\subsection{Horner's method}
An algorithm for polynomial evaluation\\
\begin{equation}
\begin{aligned}
\op{P}(x)&=a_{0}+a_{1}x+a_{2}x^{2}+a_{3}x^{3}+\cdots +a_{n}x^{n}\\
&=a_{0}+x\bigg(a_{1}+x\Big(a_{2}+x\big(a_{3}+\cdots +x(a_{n-1}+x\,a_{n})\cdots \big)\Big)\bigg)
\end{aligned}
\end{equation}This allows the evaluation of a polynomial of degree $n$ with only $n$ multiplications and $n$ additions.
This is optimal, since there are polynomials of degree $n$ that cannot be evaluated with fewer arithmetic operations

\textbf{Problem Statement:}\\
Write a function Horner() takes three inputs an array of coefficients $\op{P}$, degree of polynomial $n$ \& input $x$ and returns $\op{P}(x)$\\% which is $\op{Horner}(\op{P},n,x)$
For $k$ testcases with coefficients $P_1[\ ], P_2[\ ],\ldots,P_n[\ ]$, degree $n_1,n_2,\ldots,n_k$ and variable $x_1,x_2,\ldots,x_n$ find $P_i(x_i)$ for each $i$ in $\{1,2,\ldots,n\}$\\
\begin{testcases}
    {$k$ (number of testcases)\\$n_i$ (degree of polynomial), $x$ and $P_i[\ ]$ ($n_i+1$ coefficients with $i$\textsuperscript{th} index coefficient of $i$\textsuperscript{th} power)}
    {$\op{Horner}(\op{P},n,x)$ (each result on a newline)}
    {5\\0\quad1\quad1\\1\quad2\quad-3 2\\2\quad2\quad15 -8 7\\3\quad3\quad2 -1 -3 4\\6\quad5\quad21 10 19 47 48 9 27}
    {1\\1\\27\\80\\486421}
\end{testcases}
\subsection{Large Factorials}
\textbf{Problem Statement:}\\
Compute factorial of large numbers $n>20$
\begin{note}
\verb|long long int| can not store more than 20 digits. Maximum digits in testcases $=150$
\end{note}
\begin{testcases}
    {$n$ (number of testcases)\\[0.2em]$k_i$ for $i = 1,2,\ldots,n$\\}
    {$\op{factorial}(k_i)$ (each result on a newline)\\}
    {6\\[0.3em]21 33 57 69 77 93\\}
    {\tiny{51090942171709440000\\8683317618811886495518194401280000000\\40526919504877216755680601905432322134980384796226602145184481280000000000000\\171122452428141311372468338881272839092270544893520369393648040923257279754140647424000000000000000\\145183092028285869634070784086308284983740379224208358846781574688061991349156420080065207861248000000000000000000\\1156772507081641574759205162306240436214753229576413535186142281213246807121467315215203289516844845303838996289387078090752000000000000000000000}}
\end{testcases}
\clearpage
\subsection{Remove Duplicates}
\textbf{Problem Statement:}\\
Input an integer array and output all unique elements of that array (for repeated elements keep only first occurence)\\
\begin{testcases}
    {$k$ (number of testcases)\\$n_i$ (size of array) followed by $n_i$ elements ($<=20$) of array (each testcase on a newline)}
    {Each result on a newline}
    {2\\15\quad1 4 9 10 18 1 4 9 16 17 6 10 11 13 17\\20\quad1 3 4 8 16 1 11 15 17 20 1 3 14 15 18 4 5 17 19 20}
    {1 4 9 10 18 16 17 6 11 13\\1 3 4 8 16 11 15 17 20 14 18 5 19}
\end{testcases}
\subsection{Maximum Element}
\textbf{Problem Statement:}\\
Given an array of integers which is initially increasing and then decreasing, find the maximum value in the array.\\
A naive approach is of $\Theta(n)$ time complexity, try to think an algorithm which runs in $\Theta(\log{n})$ time (assume strictly increasing and strictly decreasing for this case)\\
\begin{testcases}
    {$k$ (number of testcases)\\$n_i$ (size of array) followed by $n_i$ distinct elements of array (each testcase on a newline)}
    {Max element of each array (each result on a newline)}
    {2\\8\quad1 4 9 10 18 17 13 11\\9\quad8 13 25 87 167 235 454 512 32}
    {18\\512}
\end{testcases}
\subsection{Majority Element}
\textbf{Problem Statement:}\\
Write a function which takes an array and prints the majority element (if it exists), otherwise prints -1.\\
A majority element in an array $A[ \ ]$ of size $n$ is an element that appears more than $n/2$ times\\%(and hence there is at most one such element)\\
A naive approach is of $\Theta(n^2)$ time complexity, try to think an algorithm which runs in $\Theta(n\log{n})$ time.\\
Can you do it in $\Theta(n)$ time?\\%Moore’s Voting Algorithm
\begin{testcases}
    {$k$ (number of testcases)\\$n_i$ (size of array) followed by $n_i$ elements of array (each testcase on a newline)}
    {Max element of each array (each result on a newline)}
    {2\\9\quad3 3 4 2 4 4 2 4 4\\8\quad3 3 4 2 4 4 2 4}
    {4\\-1}
\end{testcases}
\subsection{Matrix Inversion}
\textbf{Problem Statement:}\\
Write a function which finds multiplicative inverse of a square matrix (if it exist) using Gaussian Elimination\\
Each element of your output should contain exactly 2 decimal places\\
\begin{testcases}
    {$k$ (number of testcases)\\$n$ (size of matrix) followed by $n \times n = n^2$ elements of array}
    {Inverted Matrix}
    {3\\\textbf{2}\\4 7\\ 2 6\\\textbf{3}\\2 9 2\\3 7 1\\1 1 1\\\textbf{5}\\1 --4 3 --4 --4\\--10 4 0 7 --3\\--5 4 --5 0 --3\\9 --2 --4 --7 1\\8 --4 3 --7 5}
    {0.60 --0.70\\--0.20 0.40\\[1em]--0.43 0.50 0.36\\0.14 0.00 --0.29\\0.29 --0.50 0.93\\[1em]0.78 17.89 --5.56 11.78 5.67\\1.69 39.93 --11.70 25.35 12.61\\1.04 22.19 --6.74 14.37 7.22\\--0.19 -2.93 0.70 --1.85 --1.11\\--0.78 --14.89 4.56 --9.78 --4.67%\begin{bmatrix}0.60 &-0.70\\-0.20 &0.40\end{bmatrix}
    }
\end{testcases}