\section{Hints}
\subsection{Practice Problems 1}
\begin{figure}[h!]
\centering
\includegraphics[width=\linewidth]{HC.jpg}
\caption{Hilbert Curve Transformations}
\end{figure}
\subsection{Practice Problems 2}
\begin{enumerate}
\item
The sequence repeats when we encounter `0' then `1'
\item
Similar to calculating $e$ (problem given in slides)
\end{enumerate}
\subsection{Practice Problems 3}
\begin{enumerate}
\item
Try skipping days, months, years and centuries :D

Is there a formula?
\item
Similar pattern as in Stern Brocot Tree. Also, exactly same pattern between 3 consecutive terms
\begin{figure}[h!]
\centering
\begin{subfigure}{0.35\linewidth}
\centering
\includegraphics[width=\linewidth]{Calkin–Wilf Tree.pdf}
\caption{Calkin–Wilf Tree}
\end{subfigure}
\vline
\begin{subfigure}{0.45\linewidth}
\centering
\includegraphics[width=\linewidth]{Stern Brocot Tree.pdf}
\caption{Stern Brocot Tree}
\end{subfigure}
\vline
\begin{subfigure}{0.15\linewidth}
\centering
\includegraphics[width=\linewidth]{Thue Morse Binary Digit Sum.pdf}
\caption{Thue Morse}
\end{subfigure}
\caption{}
\end{figure}
\item 
How will you calculate $n$\textsuperscript{th} term?
\end{enumerate}
\subsection{Practice Problems 4}
Straightforward implementation of given recursive definition in problems \ref{itm:af} \& \ref{itm:part}

For \ref{itm:gp}, \ref{itm:gn} \& \ref{itm:part} a) try to find a recursive definition using previous terms\\
Observe carefully to find patterns in figures
\subsection{Practice Problems 5}
\begin{enumerate}
\item Straightforward
\item How will you store the answer? So, how can you arrive at final answer?
\item A duplicate element occurs more than one time. So, you can count occurence of each element
\item Do you really need to check all elements? Can you skip some of them?
\item Easy to solve in $\Theta(n^2)$ but there is a $\Theta(n)$ solution. You can traverse through all elements in $\Theta(n)$ time. So, how to find majority element while traversing?
\item Good Exercise. Straightforward if you know Gaussian Elimination, else you can also try other approach.
\end{enumerate}
\subsection{Practice Problems 6}
\begin{enumerate}
\item Recursive approach like \ref{itm:part}. How can you find answer for a $n$ using answers for lower $n$?
\item Divide and Conquer Approach
\item Does it feel similar to \ref{itm:dn}?
\item How to calculate first element of decompositon? After subtracting first element what will you do?
\end{enumerate}